%% Ein einfaches template für einen Übungsbericht unter verwendung des Hagenberg Setups
%% basierend auf der LaTeX 'report' Standardklasse.
%% äöüÄÖÜß  <-- no German Umlauts here? Use an UTF-8 compatible editor!

%%% Magic Comments zum Setzen der korrekten Parameter in kompatiblen IDEs
% !TeX encoding = utf8
% !TeX program = pdflatex 
% !TeX spellcheck = de_DE
% !BIB program = biber

\documentclass[german,notitlepage,smartquotes]{hgbreport}

\RequirePackage[utf8]{inputenc}		% bei der Verw. von lualatex oder xelatex entfernen!

\RequirePackage{datenumber}
\RequirePackage{stringstrings}
\RequirePackage{xstring}				% will replace stringstrings?


\renewcommand{\chapter}[1]{}		% \chapter Befehl ist deaktiviert

\graphicspath{{images/}}	% Verzeichnis mit Bildern und Grafiken

\bibliography{references}	% Biblatex-Literaturdatei (references.bib)
\ExecuteBibliographyOptions{backref=false}




%%-----------------------------------------------------------
\setcounter{chapter}{3}	% <----- Auf die Übungsnummer setzen
%%-----------------------------------------------------------

\author{Peter A.\ Schlaumeier}
\title{MTD128 Digitale Medientechnik II -- SS 2017\\
				Übungsabgabe \arabic{chapter}}
\date{\today}


%%%----------------------------------------------------------
\begin{document}
%%%----------------------------------------------------------
\maketitle
%%%----------------------------------------------------------


\verb!\hgbDate! = \hgbDate.

\subsection*{Initialization with \texttt{datetime2}}

\DTMsavedate{HGBdate}{2019-12-01}

\noindent
\DTMusedate{HGBdate}\\
\DTMfetchyear{HGBdate}\\
\DTMfetchmonth{HGBdate}\\
\DTMfetchday{HGBdate}\\




\subsection*{Calculations with package \texttt{datenumber}}

\setstartyear{2000}
\setdatenumber{2007}{11}{22}
\noindent
datenumber = \thedatenumber.

\newcounter{mydatenumber}
\setmydatenumber{mydatenumber}{2007}{11}{23}
\noindent
mydatenumber = \themydatenumber.

%% -----------------------

\newcounter{HGBdatenumber}
\setmydatenumber{HGBdatenumber}{\DTMfetchyear{HGBdate}}{\DTMfetchmonth{HGBdate}}{\DTMfetchday{HGBdate}}
\noindent
HGBdatenumber = \theHGBdatenumber.

\newcounter{dateHgb}%
\newcounter{dateAge}%

%\newcommand{\daydifftoday}{%
\setmydatenumber{dateHgb}{\DTMfetchyear{HGBdate}}{\DTMfetchmonth{HGBdate}}{\DTMfetchday{HGBdate}}%
\setmydatenumber{dateAge}{\the\year}{\the\month}{\the\day}%
\addtocounter{dateAge}{-\thedateHgb}%
%\thedateAge
%}

\noindent
HagenbergThesis is \thedateAge\ days old.

\newcounter{ageLimit}\setcounter{ageLimit}{6}

\noindent
\ifnum\value{dateAge}>\value{ageLimit}%
     HGB is too old (limit is \arabic{ageLimit})!%
    \else 
      HGB is fine.%
    \fi

\PackageWarning{hagenberg-thesis}{%
^^J***********************************************************
^^J'hagenberg-thesis' is \arabic{dateAge} days old - pls. check for updates at
^^Jhttps://github.com/Digital-Media/HagenbergThesis
^^J***********************************************************
^^J} % https://tex.stackexchange.com/questions/6529/newline-linebreak-in-message

%% ----------------------------------------------------------------------------------------





\subsection*{Problem: convert from ``2014/12/20'' (Latex package version date) to \texttt{DTMdate}}

\noindent
\textbf{Look at} \url{https://tex.stackexchange.com/questions/396363/input-a-date-to-datetime2-in-a-format-different-than-iso/396483#396483}
for adapting to Latex date format!

%% problem: convert from "2014/12/20" -> DTNdate

%\url{https://tex.stackexchange.com/questions/218813/latex-access-every-character-of-an-string-variable
%\usepackage{stringstrings}

%\def\x{2014/12/20}
\newcommand{\hgbDateXX}{2019/12/10}
\def\SSyear{\substring{\hgbDateXX}{1}{4}}
\def\SSmonth{\substring{\hgbDateXX}{6}{7}}
\def\SSday{\substring{\hgbDateXX}{9}{10}} % The $ implies the last character of the string

\noindent
\verb!\SSyear\ \SSmonth\ \SSday! = \SSyear\ \SSmonth\ \SSday

\newcommand{\MakeDateString}[3]{%
#1-#2-#3
}

\def\DateString{\MakeDateString{\SSyear}{\SSmonth}{\SSday}}

\noindent
MakeDateString1: \MakeDateString{\SSyear}{\SSmonth}{\SSday}\\
MakeDateString2: \DateString

% Look here: https://tex.stackexchange.com/questions/335635/expansion-of-argument-to-dtmdate-function-in-datetime2 !!
% https://tex.stackexchange.com/questions/320256/datetime2s-dtmsavetimestamp-cant-read-from-macro-expansion-or-catcode-issue?rq=1
%\DTMsavedate{HgbBuildDateA}{\DateString}                   % not working!
%\DTMsavedate{HgbBuildDateA}{\SSyear-\SSmonth-\SSday}       % not working!
\DTMsavenoparsedate{HgbBuildDateB}{2019}{12}{10}{-1} %{\SSyear}{\SSmonth}{\SSday}{-1}

\def\YYYY{2019}
\def\MM{12}
\def\DD{10}
\DTMsavenoparsedate{HgbBuildDateC}{\YYYY}{\MM}{\DD}{-1}     % working!
%\DTMsavenoparsedate{HgbBuildDateC}{\SSyear}{\SSmonth}{\SSday}{-1}     % NOT working!
\noindent
HgbBuildDateC = \DTMusedate{HgbBuildDateC}


%% -------------------------------------------------------
%https://tex.stackexchange.com/questions/321313/change-pre-defined-date-in-command-and-save-to-datetime2-expansion-issue?rq=1
% THIS IS THE SOLUTION!!!!???

\def\eDTMsavedate#1#2{% #1: counter-name, #2: date-string
  \begingroup\edef\@x{\endgroup
    \noexpand\DTMsavedate{#1}{#2}%
  }\@x
}

\def\origDate{1977-10-26}
\eDTMsavedate{origdate}{\origDate}%
\noindent
origdate = \DTMusedate{origdate}


%%%----------------------------------------------------------

% simpler calculation (without datenumber package)

\DTMsavenoparsedate{TheDateA}{2019}{12}{01}{-1}
\DTMsavenoparsedate{TheDateB}{2019}{12}{10}{-1}

\newcount\myDiff
\DTMsaveddatediff{TheDateB}{TheDateA}{\myDiff}

\noindent
myDiff = \the\myDiff.

\noindent
\ifnum\myDiff>6
    HGB is too old (limit is \arabic{ageLimit})!%
\else
    HGB is fine.%
\fi


\subsection*{Testing \texttt{xstring} package}

%% see https://tex.stackexchange.com/questions/451/when-to-use-edef-noexpand-and-expandafter

\expanded{\YYYY-}

%\edef\EDEFSTRING{\SSyear}  %{\SSyear-\SSmonth-\SSday}
%\verb!\EDEFSTRING! = \EDEFSTRING

%\newcommand{\YearD}[0]{\StrLeft{\SSyear\-\SSmonth\-\SSday}{10}}
%\verb!\YearD! = \StrLeft{\SSyear-\SSmonth-\SSday}{10}

%\DTMsavenoparsedate{TheDateD}{1977}{12}{10}{-1}



\end{document}
