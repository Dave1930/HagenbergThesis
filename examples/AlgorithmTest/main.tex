%% A simple template for a term report using the Hagenberg setup
%% based on the standard LaTeX 'report' class
%% äöüÄÖÜß  <-- no German Umlauts here? Use an UTF-8 compatible editor!

%%% Magic comments for setting the correct parameters in compatible IDEs
% !TeX encoding = utf8
% !TeX program = pdflatex 
% !TeX spellcheck = en_US
% !BIB program = biber

\documentclass[notitlepage,english]{hgbreport}

\RequirePackage[utf8]{inputenc}		% remove when using lualatex oder xelatex!
\RequirePackage{hgbalgo}

%\graphicspath{{images/}}  % where are the images?
%\bibliography{references}  % requires file 'references.bib'

%%%----------------------------------------------------------
\author{Peter A.\ Wiseguy}
\title{CS799 Ridiculously Advanced Systems\\ % the name of the course or project
			Project Report}	% or "Term Report"
\date{\today}
%%%----------------------------------------------------------


%%%----------------------------------------------------------
\begin{document}
%%%----------------------------------------------------------

\maketitle

\begin{abstract}\noindent
This document is a simple template for a typical term or semester paper (lab/course report, 
``Übungsbericht'', \etc) based on the \textsf{HagenbergThesis} \latex package.%
\footnote{See \url{https://github.com/Digital-Media/HagenbergThesis} for the most current version
and additional examples.
This repository also provides a good introduction and useful hints for authoring academic texts with LaTeX.}
The structure and chapter titles have been formulated to provide a good starting point
for a typical \emph{project report}.
This document uses the custom class \textsf{hgbreport} which is based on \latex's standard \textsf{report} 
document class with \texttt{chapter} as the top structuring element. 
If you wish to write this report in German you should substitute the line
\begin{quote}
	\verb!\documentclass[english]{hgbreport}! 
\end{quote}
at the top of this document by
\begin{quote}
	\verb!\documentclass[german]{hgbreport}!.
\end{quote}
To omit the default \textbf{title page} (as in this document) use the \texttt{notitlepage} option, \eg,
\begin{quote}
	\verb!\documentclass[notitlepage,english]{hgbreport}!.
\end{quote}
Also, you may want to place the text of the individual chapters in separate files and 
include them using \verb!\include{..}!.

\bigskip
\noindent
Use the abstract to provide a short summary of the contents in the document.
\end{abstract}


%%%----------------------------------------------------------
%\tableofcontents
%%%----------------------------------------------------------



%%%----------------------------------------------------------
\chapter{Algorithm test}
%%%----------------------------------------------------------


\show\State

%%--------------------------------------------------------------------

\begin{algorithm}
\caption{Bikubische Interpolation in 2D.
	Die Funktion $w_{\mathrm{cub}}()$ steht für die 
	eindimensionale kubische Interpolationsfunktion (Zeile~\ref{alg:wcub}).}
\label{alg:Example}

\begin{algorithmic}[1]     % [1] = all lines are numbered
\Procedure{BicubicInterpolation}{$I, x, y$} \Comment{$x,y \in \R$}
	\Statex \textbf{Input:} 
	\Statex Returns the interpolated value of the image $I$ at the continuous position $(x, y)$.
	
	\State $\mathit{val} \gets 0$
	
	\For{$j \gets 0, \ldots, 3$} \Comment{iterate over 4 lines}
		\State $v \gets \lfloor y \rfloor - 1 + j$
		\State $p \gets 0$
		
		\For{$i \gets 0, \ldots, 3$} \Comment{iterate over 4 columns}
			\State $u \gets \lfloor x \rfloor - 1 + i$
			\State $p \gets p + I(u,v) \cdot w_{\mathrm{cub}}(x - u )$
					\label{alg:wcub}
		\EndFor
		
		\State $\mathit{val} \gets \mathit{val} + p \cdot w_{\mathrm{cub}}(y - v)$
	\EndFor
	
	\State\Return $\mathit{val}$
	
\EndProcedure
\end{algorithmic}
\end{algorithm}

%%--------------------------------------------------------------------



\begin{algorithm}	 %%% Preindl
\caption{Finds a minimum makespan role assignment. This function is the MMDR $O(n^5)$ polynomial time implementation, as described by McAlpine et al. It rearranges target positions $T$ so that their index corresponds with the indices of their assigned agents.}
\label{alg:mmdr}
\begin{algorithmic}[1]% [1] = all lines are numbered
\Procedure{RoleAssignment}{$A,T$}
%
\Input $A = (\mathbf{a}_0, \dots, \mathbf{a}_{n-1})$, 
	$T = (\mathbf{t}_0, \dots, \mathbf{t}_{n-1})$,
	where $\mathbf{a}_i = (x_i, y_i)$, $\mathbf{t}_j = (x_j, y_j)$. 
\Output $T' = (\mathbf{t}'_0, \dots, \mathbf{t}'_{n-1})$, a permutation of $T$.
%
\StateNN[1] StateNN[1]
\State State
\StateNN StateNN (no argument)
\StateNN[1] StateNN[1]
\StatexIndent[1] StatexIndent
\State State
\Statex Statex
\StateY StateY
%\StateZ StateZ
\item[] item[]
\LineComment{no need to explore more. we just want to stop over here.}


\State $\mathit{edgesSorted} \gets \Call{SortAscendingDist}{\mathit{Edges}}$
\State $\mathit{lastDistance} \gets -1$
\State $\mathit{rank} \gets 0$
\State$\mathit{currentIndex} \gets 0$


\Statex Process sorted edges (INDENTATION PROBLEM): \SHOWnested
\For{$e \in \mathit{edgesSorted}$}
	\LineComment{no need to explore more. we just want to stop over here. \SHOWnested}
	\If{$||e|| > \mathit{lastDistance}$} \SHOWthistlm\ \SHOWnested
		\State State \SHOWthistlm\ \SHOWnested
		\LineComment{no need to explore more \SHOWthistlm}
		\StateY StateY
		%\StateZ StateZ
		\StateNN StateNN
		
		\StatexIndent[3] StatexIndent
		
		%\StateAAA{StateAAA}
		\Statex Statex \SHOWnested
		\item[] item[] \SHOWnested
		
		\State$\mathit{rank} \gets \mathit{currentIndex}$ \SHOWnested
	\EndIf
	\State$\mathit{lastDistance} \gets ||e||$ \SHOWnested
	\State$||e|| \gets 2^{\mathit{rank}}$
	\State$\mathit{currentIndex} \gets \mathit{currentIndex} + 1$
	\State finally \SHOWnested
\EndFor

\State $\mathit{perfectMatching} \gets \Call{HungarianAlg}{\mathit{edgesSorted}}$ \Comment{returns a set of edges}
\EndProcedure
\end{algorithmic}
 \end{algorithm}



\end{document}
