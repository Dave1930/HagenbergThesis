\chapter{Einleitung}
\label{cha:Einleitung}

\section{Zielsetzung}
Dieses Dokument ist als vorwiegend technische Starthilfe für das
Erstellen einer Masterarbeit (oder Bachelorarbeit) mit \latex
gedacht und ist die Weiterentwicklung einer früheren
Vorlage\footnote{Nicht mehr verfügbar.} für das Arbeiten mit
Microsoft \emph{Word}. Während ursprünglich daran gedacht war, die
bestehende Vorlage einfach in \latex zu übernehmen, wurde rasch
klar, dass allein aufgrund der großen Unterschiede zum Arbeiten
mit \emph{Word} ein gänzlich anderer Ansatz notwendig wurde. Dazu
kamen zahlreiche Erfahrungen mit Diplomarbeiten in den
nachfolgenden Jahren, die zu einigen zusätzlichen Hinweisen Anlass gaben.

Das vorliegende Dokument dient einem zweifachen Zweck: 
\emph{erstens} als Erläuterung und Anleitung, \emph{zweitens} als
direkter Ausgangspunkt für die eigene Arbeit. Angenommen wird,
dass der*die Leser*in bereits über elementare Kenntnisse im Umgang mit
\latex verfügt. In diesem Fall sollte -- eine einwandfreie
Installation der Software vorausgesetzt -- der Arbeit nichts mehr
im Wege stehen. Auch sonst ist der Start mit \latex\ nicht
schwierig, da viele hilfreiche Informationen auf den zugehörigen
Webseiten zu finden sind (s.\ Kap.~\ref{cha:ArbeitenMitLatex}).





\section{Warum {\latex}?}

Diplomarbeiten, Dissertationen und Bücher im
technisch-natur\-wissen\-schaft\-lichen Bereich werden
traditionell mithilfe des Textverarbeitungssystems \latex
\cite{Lamport1994, Lamport1995} gesetzt. Das hat gute Gründe, denn
\latex ist bzgl.\ der Qualität des Druckbilds, des Umgangs mit
mathematischen Elementen, Literaturverzeichnissen etc.\
unübertroffen und ist noch dazu frei verfügbar. Wer mit \latex
bereits vertraut ist, sollte es auch für die Abschlussarbeit
unbedingt in Betracht ziehen, aber auch für den Anfänger sollte
sich die zusätzliche Mühe am Ende durchaus lohnen.

Für den professionellen elektronischen Buchsatz wurde früher
häufig \emph{Adobe Framemaker} verwendet, allerdings ist diese
Software teuer und komplex. Eine modernere Alternative dazu ist
\emph{Adobe InDesign}, wobei allerdings die Erstellung
mathematischer Elemente und die Verwaltung von Literaturverweisen
zur Zeit nur rudimentär unterstützt werden.%
\footnote{Angeblich werden aber für den (sehr sauberen) Schriftsatz 
in \emph{InDesign} ähnliche Algorithmen wie in \latex\ verwendet.}

Microsoft \emph{Word} gilt im Unterschied zu \latex, 
\emph{Framemaker} und \emph{InDesign} übrigens nicht als professionelle
Textverarbeitungssoftware, obwohl es immer häufiger auch von
großen Verlagen verwendet wird.%
\footnote{Siehe auch \url{https://latex.tugraz.at/dokumentation/mythen}.}
Das Schriftbild in \emph{Word}
lässt -- zumindest für das geschulte Auge -- einiges zu wünschen
übrig und das Erstellen von Büchern und ähnlich großen Dokumenten
wird nur unzureichend unterstützt. Allerdings ist \emph{Word} sehr
verbreitet, flexibel und vielen Benutzer*innen zumindest oberflächlich
vertraut, sodass das Erlernen eines speziellen Werkzeugs wie
\latex\ ausschließlich für das Erstellen einer Abschlussarbeit
manchen verständlicherweise zu mühevoll ist. Es sollte daher
niemandem übel genommen werden, wenn er*sie sich auch bei der Abschlussarbeit
auf \emph{Word} verlässt. Im Endeffekt lässt sich mit etwas
Sorgfalt (und ein paar Tricks) auch damit ein durchaus akzeptables
Ergebnis erzielen. 
Ansonsten sollten auch für \emph{Word}-Benutzer*innen 
einige Teile dieses Dokuments von Interesse sein, insbesondere die
Abschnitte über Abbildungen und Tabellen
(Kap.~\ref{cha:Abbildungen}) und mathematische Elemente
(Kap.~\ref{cha:Mathematik}).


\section{Aufbau der Arbeit}

Hier am Ende des Einleitungskapitels (und nicht
etwa in der Kurzfassung) ist der richtige Platz, um die
inhaltliche Gliederung der nachfolgenden Arbeit zu beschreiben.
Hier sollte man darstellen, welche Teile (Kapitel) der Arbeit
welche Funktion haben und wie sie inhaltlich zusammenhängen. Auch
die Inhalte des \emph{Anhangs} -- sofern vorgesehen -- sollten hier
kurz beschrieben werden.

Zunächst sind in Kapitel \ref{cha:Abschlussarbeit} einige wichtige
Punkte zu Abschlussarbeiten im Allgemeinen zusammengefasst.
Kapitel \ref{cha:ArbeitenMitLatex} beschreibt die Idee und die
grundlegenden technischen Eigenschaften von \latex.
Kapitel \ref{cha:Abbildungen} widmet sich der Erstellung von Abbildungen
und Tabellen sowie der Einbindung von Quellcode.
Mathematische Elemente und Gleichungen sind das Thema in Kapitel \ref{cha:Mathematik} 
\usw
Anhang \ref{app:TechnischeInfos} enthält technische Details zu
dieser Vorlage, 
Anhang \ref{app:materials} enthält eine Auflistung von zugehörigen Materialien
auf einem beigelegten Speichermedium, und 
Anhang \ref{app:Fragebogen} zeigt ein Beispiel für die
Einbindung eines mehrseitigen PDF-Dokuments.






